% macros.tex

%%%
% This document contains maxcros used in your thesis.
%
% It is a VERY important that you use macros to establish your notation
% In particular:
%  1.  Use macros for acronyms -- to make sure they are expanded properly@
%  2.  Use macros for jargon / terms in your field.
%      This is important so you can use a spell-checker!
%      It also is helpful if you change the way you render important words.
%  3.  Use macros for math symbols and notation.
%      - It is EXTREMELY common to change the names of variable or functions
%        or set etc.
%      - You WILL make mistakes if you just try to duplicate the same
%        latex everywhere.

%  This is just like codeing -- break the document up into manageable functions!!


% Example: The expected value
%  Renders as E[X] -- the expected value.
\def\Expected[#1]{\ensuremath{\mathbb{E}\left[{#1}\right]}}

% Just renders as n -- but I may want to change it to N or k or K later....
\def\NumberOfRecords{\ensuremath{n}}

% Puts a bulleted list item in a table
\newcommand{\tabitem}{~~\llap{\textbullet}~~}

% Using the glossaries package to manage acronyms and jargon
\newacronym{html}{HTML}{hypertext markup language}
\newacronym{xml}{XML}{extensible markup language}
\newacronym{css}{CSS}{cascading style sheet}

\newacronym{dry}{DRY}{`don't repeat yourself'}
\newacronym[see={apig}]  % Cross-reference to glossary
    {api} % label with \gls{api},\glspl{api}, \Gls{api},\Glspl{api}
    {API} % short form
    {Application Programming
        Interface\glsadd{apig}} % Use glsadd to force glossaryentry to shoew


% Math symbols used
\glsxtrnewsymbol[
    see={real number},
    description={The set of \glspl{real number}} % Description
    % Other options can go here to....
    ]
    {real-number-symbol} % label
    {$\mathbb{R}$}       % symbol

% Jargon / Terms
\newglossaryentry{apig}{
    name={API},
    see={api},
    description={
        An Application Programming Interface (API) is a particular set
        of rules and specifications that a software program can follow to access and
        make use of the services and resources provided by another particular software
        program that implements that API
        }
}



\newglossaryentry{real number}
{
    name={real number},
    see={real-number-symbol},
    description={Includes both rational numbers, such as $42$ and
        $\frac{-23}{129}$, and irrational numbers,
        such as $\pi$ and the square root of two; or,
        a real number can be given by an infinite decimal
        representation, such as $2.4871773339\ldots$ where
        the digits continue in some way; or, the real
        numbers may be thought of as points on an infinitely
        long number line},
    symbol={\ensuremath{\mathbb{R}}}
}