\chapter{Introduction}
\label{chap:introduction}

 Start with the \emph{need}.
 There is a \emph{gap} in our current capabilities -- show the gap before you propose to fill it.

 This is the introduction of the entire thesis.
 The target audience is a reader who is NOT an expert in your focus area.
 You need to convince them that your TOPIC is important but you can not get too technical
 yet.  This is the Layman's version of your problem.

 Start by telling a story:  what are some important or excting tasks that the reader might
  wish to ba able to do.
  The goal is to convince the reader that the WANT or NEED something before you reveal that
  you are gooing to get it for them.

 If you are having a hard time generating ideas on what capabilities to list,  do this:

\begin{enumerate}
    \item    Go through all of the most similar related work you can find on your topic.
   \item  Look at the first paragraphs of their introduction -- they almost certainly
       list some application of their work.   Write doewn their applications as comments
       in this document.  Then try to cherry-pick the examples that are most relevant to
       your own work.  Do not copy what they say word for word -- but you can learn a lot
       from the way competing papers sell their work.
\end{enumerate}


  Once you have listed some needed capabilities, say that they all boil down to one important
  problem that needs to be solved.



I have used the \emph{glossaries} package to manage acronyms and terms in the thesis.
You \textbf{should use }  this. I do not like reading long documents and doing a scavenger hunt to see what a symbol means! I provide example entries for \glspl{real number} \glsadd{real-number-symbol} and \gls{api}.

\section{Problem Statement} \label{sec:problem-statement}
We describe the XXX problem as follows: ...

 All problem statements include the following:
 \begin{description}
     \item[GIVEN:] What is the \emph{input} or what are the \emph{initial conditions}
   \item[SUCH THAT:] What is the scope? What are the limitations on the input that
                you need to impose in order to be able to solve it?
   \item[AIM:] What is your \emph{goal}?   What do you want to accomplish?
   \item[CRITERA:]  How can you measure whether you have done a good job? This needs to
               be something that you can measure. It is possible to use even
               subjective things like "is this fake image realistic" by using
               a user study -- in that case criteria would be that
               a human user cannot distinguish real from fake images X\% of the time.
 \end{description}
 It is up to you whether you call out those parts or just use POE (plain old English)

  You will \emph{probably} want to refine your problem statement to include several other
  sub-problems that need to be solved.  These can form the basis of multiple contributions
  or chapters later in the work.

\section{Contributions} \label{sec:contributions}
The thesis of this work is ....
 re-word contribution \ref{itm:first}, it will be your thesis.

In particular, we
expect   %<-- comment this line out for the final thesis.
% make   %<-- uncomment this for the final Thesis (after your proposal)
the following contributions:
\begin{enumerate}
    \item \label{itm:first} List your main, most important contribution here. This is the "real" one
    \item  List a secondary contribution. This one is likely less signficant than the
           first one, but DO NOT admit that to ANYONE
    \item  And list your weakest one here. Always good to have three -- try to make sure
           you have 3 plausible things -- but the first one should be the winner
\end{enumerate}


% The last paragraph explaines the stryucture of the rest of the Thisis.
The rest of the thesis is organized as follows....



