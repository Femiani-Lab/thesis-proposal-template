\chapter{Background \& Related Work}
\label{chap:background}

\section{Introduction}
\label{sec:background:introducton}
This section is where you will discuss relevant background work, and related works for comparison. Ensure that you cite references appropriately, using this as an example~\cite{sample2019}

Dr. Femiani wants you to read examples of published literature reviews (Google for special journals deveoted to them).  Your aspiration or goal should be to write a lit. review as a standalone document that is capable of being published.

 Some guidelines:

\begin{enumerate}
    \item  Put the \verb|\cite{}| as close as possible to the \emph{first} mention of the work.
     \emph{Do not} put it at the end of the sentence, but instead put it right after you \emph{first} mention the work.

     Do this:
\begin{quote}
         \verb|The FOO Method~\cite{authors} used whozits and whatsits to do stuff.|
\end{quote}
     and not this:
    \begin{quote}
         \verb|The FOO Method used whozits and whatsits to do stuff~\cite{authors}.|
    \end{quote}
   An exception is if you are using a quote -- but then you also have to
   use quotes(!).

   Also when possible provide a speak-able name for a paper in addition to
   a citation. You can say
   \begin{verbatim}
  the method of \cite{authors} was blah blah
   \end{verbatim}
   but if I prefer to say
\begin{verbatim}
   the method of FOO~\cite{authors}.
\end{verbatim}
   This is controversial.

\item  You can put multiple bibetex keys in one \verb|\cite{}|, and you should
\item Use \verb|\cite[p.~150]{sample2019}| to cite a specific page.
\item Always put citations in figure captions if you copy them
\item But you should mostly make your own figures (except maybe when showing others' results in background section)
\end{enumerate}

Start by explaining:
\begin{enumerate}

    \item Why should I read this?
    \begin{itemize}
        \item Do you include new material not yet covered in a lit review?
        \item Do you discover new connections between past work?
        \item Do you cover it in a new way by giving a new interpretation?
        \item Can you identify an emerging trend or anticipate future directions?
        \item Do you identify gaps in the research that are ready to be explored? I hope so!
    \end{itemize}
    \item How did you select the topics to cover.
    \item What is the criteria for something to be included
    \item How is the chapter organized. Prefer to organise according to themes or methods as the demonstrates greater analytical abilities.
\end{enumerate}



\section{Overview}

My preference is to use a figure such as a concept map to show how the work is connected. The figure includes sections numbers. Use InkScape (\url{https://inkscape.org/}) end export to PDF with LaTeX to make such a figure.
 See the discription of how to do this\footnote{Putting SVG figuresinto \LaTeX{} is described at \url{http://tug.ctan.org/tex-archive/info/svg-inkscape/InkscapePDFLaTeX.pdf}.  }
 In fact I would go so fat as to say \textbf{You should \textit{start} with a concept map and a table - Use tools like freemind to build them while you collect your sources }


It is also a good idea to organize the work according to different features.
Then present a table like \autoref{tab:work-compared}.   If  you end up with a very big table, you can print it in landscape mode like \autoref{tab:big-work-compared} on \autopageref{tab:big-work-compared}.

\begin{table}[htbp]
    \centering
 \begin{threeparttable}
    \caption[Short Title for List of Table]{A comparison of methods}
    \label{tab:work-compared}
    \begin{tabular}{lllll}
        Approach                          & Feature & Feature & Feature & Feature \\ \hline
        Method A~\cite{sample2019}&         &         &         &         \\
        Method B~\cite{sample2019}&         &         &         &         \\
        Method C~\cite{sample2019} &         &         &         &         \\ \hline
    \end{tabular}
    \begin{tablenotes}
        \small
        \item This is where authors provide additional information about
        the data, including whatever notes are needed.
        \item It is good for Dr. Femiani, who does not read the text first.
    \end{tablenotes}
\end{threeparttable}
\end{table}


\begin{landscape}
\begin{table}[]
    \centering
    \begin{threeparttable}
        \caption[Short Title for List of Table]{A really wide comparison of methods}
        \label{tab:big-work-compared}
        \begin{tabular}{lllll}
            Approach                   & Feature & Feature & Feature & Feature \\ \hline
            Method A~\cite{sample2019}
            & %Featur1
            {\small Yes, Explain}
            &% Feature2
            {\small Partially, Explain }
            &% Feature 3
            {\small No, Explain}
            &% Feature 4
            {\small NA} \\
            Method B~\cite{sample2019} &         &         &         &         \\
            Method C~\cite{sample2019} &         &         &         &         \\ \hline
        \end{tabular}
        \begin{tablenotes}
            \small
            \item This is where authors provide additional information about
            the data, including whatever notes are needed.
            \item It is good for Dr. Femiani, who does not read the text first.
        \end{tablenotes}
    \end{threeparttable}
\end{table}
\end{landscape}



\section{Background Topic 1}

\section{Background Topic 2}

\section{Discussion}

Demonstrate a deep understanding of the work. Add to it by pointing out connections and relationships.

\section{Conclusion}
Concluse \textit{this chapter} by pointing out that there is still an important gap in the current state of the art.